\documentclass[twocolumn,10pt]{article}
\linespread{1.3}
\usepackage[margin=2cm]{geometry}

\usepackage[utf8]{inputenc}
\usepackage[spanish]{babel}
\usepackage{tikz}
\usepackage{paralist}
\usepackage{subcaption} \usepackage{graphicx}
\usepackage{amsmath} \usepackage{amssymb}
\usepackage{xcolor}
\usepackage{listings}
\lstset{language=C}
\bibliographystyle{apsrev4-1}

\graphicspath{{./fig/}}

\newcommand{\IDT}{\texttt{IDT}}
\newcommand{\MMU}{\texttt{MMU}}
\newcommand{\GDT}{\texttt{GDT}}
\newcommand{\TSS}{\texttt{TSS}}
\newcommand{\PD}{\texttt{PD}}

\begin{document} 

\title{Jaurías: Programación de Sistemas Operativos}

\author{M. I. Gamarra, D. Santos y P. N. Alcain}

% \affiliation{\textbf{Grupo:} My Generation}

\date{\today}

%\begin{abstract}
%\end{abstract}

\maketitle

\section{Introducción}
El trabajo práctico consiste en la implementación de un pequeño
Sistema Operativo que dé soporte para que corra una version del juego
\textbf{Jaurías}. El desarrollo se basó en una serie de ejercicios que
sirvieron como pautas de ordenamiento para la implementación del
sistema.

Los ejercicios estaban enfocados en la implementación del SO a través
de ciertas etapas específicas para el funcionamiento del sistema.

Las etapas desarrolladas son:

\begin{itemize}
\item Tabla de Descriptores Globales (\GDT): Armamos la tabla
  \GDT para direccionar los primeros \texttt{500 MB} que
  contengan los niveles \emph{usuario} y \emph{kernel} y tanto datos
  como códigos. La segmentación es del tipo \emph{flat} (es decir,
  todos los segmentos comparten el espacio total de los \texttt{500
    MB}) ya que los permisos luego se conforman con la
  paginación.

\item Tabla de Descriptores de Interrupción (\IDT) de Procesador: Armamos la
  tabla correspondiente a \IDT para las interrupciones del
  procesador.

\item Inicialización de Directorio de Páginas (\PD): Creamos e
  inicializamos el directorio páginas y las tablas de pagina para el
  \emph{kernel} con \emph{identity mapping} para el mapeo de las
  páginas. Una vez que está mapeada la primera entrada, activamos
  paginación.

\item Unidad de Manejo de Memoria (\MMU): Escribimos una
  unidad de memoria para manejar la paginación dinámica, que mapea y
  desmapea páginas a medida que las tareas lo requieren.

\item Tabla de Descriptores de Interrupción (\IDT) de Usuario:
  Completamos la \IDT con las rutinas de atención de interrupciones
  del usuario (teclado, reloj y servicio \texttt{0x46}). Además se
  cambió el handler de las otras interrupciones para que, ante
  cualquier excepción generada por una tarea asociada a un jugador,
  ésta sea eliminada. La interrupción \texttt{0x46} es una
  \texttt{syscall} que programamos de acuerdo a las acciones de las tareas.

\item Tabla de Tareas (\TSS): Armamos la tabla de \TSS con 18 entradas
  en total (la tarea inicial, la tarea \texttt{Idle} y las tareas de
  los 8 perros para cada jugador). Luego completamos en la \GDT los
  selectores de todas las táreas. Completamos cada entrada de la \TSS
  con el código de ejecución común a los perros,
  \texttt{0x401000}. Como \texttt{cr3} cambia según el tipo de perro,
  \texttt{0x401000} está mapeado a distintas posiciones de memoria y, en
  consecuencia, ejecuta código distinto. Aquí estamos usando
  claramente la ventaja de la paginación.

\item Scheduler: Programamos un scheduler de tipo Round-Robin que
  intercambia las tareas, de modo tal de poder simular el
  procesamiento ``simultáneo'' de tareas.

\item Modo Debug: La idea era implementar esta característica del
  juego, que pone el sistema en pausa y muestra el estado de los
  registros del procesador cuando una tarea genera una excepción.

\end{itemize}

\section{El kernel}

La tarea clave del kernel es pasar de modo real a modo protegido. Una
vez en modo protegido, ya tenemos un kernel de 32 bits
funcionando. Para pasar a modo protegido necesitamos definir la \GDT.

\subsection{Ejercicio 1: \GDT}

En este ejercicio armamos la \GDT. Es simplemente un arreglo de
entradas con la siguiente estructura:

\begin{lstlisting}
  struct str_gdt_entry {
    unsigned short  limit_0_15;
    unsigned short  base_0_15;
    unsigned char   base_23_16;
    unsigned char   type:4;
    unsigned char   s:1;
    unsigned char   dpl:2;
    unsigned char   p:1;
    unsigned char   limit_16_19:4;
    unsigned char   avl:1;
    unsigned char   l:1;
    unsigned char   db:1;
    unsigned char   g:1;
    unsigned char   base_31_24;
  } __attribute__((__packed__, aligned (8)));

  /* sizeof(struct str_gdt_entry) = 8 bytes */
\end{lstlisting}

Hay un detalle específico a este TP en particular, y es que además del
\texttt{NULL\_DESCRIPTOR} en la posicion 0, recién llenamos la \GDT a
partir del índice 8. La segmentación, como ya dijimos, es de tipo
\emph{flat}, así que todas las entradas de la \GDT tienen que
direccionar los primeros \texttt{500 MB}. Para poder direccionar tanta
memoria, necesitamos que la granularidad sea 1 así direcciona de a
páginas de \texttt{4 KB}. La primera entrada de la GDT es una entrada
nula y, a partir de la octava entrada, cargamos los descriptores
correspondientes a cada nivel (0 y 3) para código y datos
respectivamente. La única diferencia entre cada par de segmentos en el
mismo nivel es únicamente el campo tipo. El límite es \texttt{500
  MB/4kb - 1= 0x1F400 - 1 = 0x1F3FF}, la última página
direccionada. Así, por ejemplo, un descriptor de segmento de código de
nivel 0 es de la forma:
\begin{itemize}
\item \textbf{límite} = \texttt{0x001F3FF} (\texttt{limit(0:15) =
    0xF3FF} y \texttt{limit(16:19) = 0x1}).
\item \textbf{base} = \texttt{0x0} (\texttt{base(0:15) = 0x0} y
  \texttt{base(16:19) = 0x0}).
\item \textbf{tipo} = \texttt{0xA} (de acuerdo a la tabla 3-1 de
  Intel. También se puede pensar en setear cada bit por separado) que
  indicar que es de código y permitir que se pueda leer y ejecutar. (o
  sea, el tipo de memoria y el privilegio)
\item En el bit de sistema le pusimos 1 para indicar que no es
  de sistema.
\item Le asignamos privilegio de kernel (DPL = 0).
\item Lo marcamos como presente.
\item En el campo AVL pusimos 0 al igual que en el campo \textit{l}
  pues trabajamos con la arquitectura de 32 bits y como trabajamos con
  segmentos de 32 bits marcamos el bit DB como uno.  
\end{itemize}


El resto de las definiciones es análoga. Una vez hecho esto,
preparamos el sistema para pasar a modo protegido. Para pasar a modo
protegido, lo que realizamos fue:
\begin{itemize}
\item Deshabilitar las interrupciones, usando \textbf{cli}.
\item Habilitamos A20 (llamando a la rutina \texttt{habilitar\_ A20}).
\item Cargamos la GDT utilizando la instrucción
  \texttt{lgdt[GDT\_ DESC]}, donde \texttt{GDT\_DESC} es la base en la
  que definimos la la GDT.
\item Seteamos el bit PE del registro CR0 realizando un OR
  entre lo que contiene este registro y \textbf{0x1}
\item Finalmente realizamos el salto a
  \texttt{(0x8*8):modo\_rotegido}, pues en la posición \textbf{0x8} de
  la GDT se encuentra el selector de código de nivel 0 y
  \texttt{modo\_protegido} es la etiqueta donde empezamos a definir el
  código correspondiente al modo protegido.
\end{itemize} 

Estando en modo protegido, seteamos los correspondientes registros de
segmento de datos y el segmento extra para video (en
\texttt{fs}). Finalmente, seteamos la pila en la dirección
\texttt{0x27000}, una vez en modo protegido, cargamos los registros
\texttt{ebp} y \texttt{esp} con el valor \texttt{0x27000}.

\subsubsection*{Específicos de este ejercicio}

Finalmente, para terminar el ejercicio, inicializamos la pantalla
llamando a la función proporcionada por la cátedra
\texttt{screen\_inicializar()} la cual pinta el área de la pantalla
con los colores que se muestran en el enunciado.

\subsection{Ejercicio 2: Interrupciones del procesador}

Completamos las entradas de la IDT para las entradas 0-19 (20 a 31
estan reservadas). Todas estas entradas fueron inicializadas con los
siguientes valores:
\begin{itemize}
\item En el campo offset(15:0) se colocan los últimos 2 bytes de la
  dirección donde se encuentra la rutina de atención de la interrupción
  en cuestión.
\item En el selector de segmento ponemos \texttt{0x0008}, que corresponde a un
  segmento de código de nivel 0.
\item En los atributos de la entrada \texttt{IDT} usuario colocamos \texttt{0xEE00}
  (macro \texttt{ID\_ENTRY}) que indica que estamos trabajando con una
  interrupt gate, de tamaño 32 bits, \texttt{DPL} 3 y presente.
\item En los atributos de la entrada \texttt{IDT} supervisor colocamos \texttt{0x8E00}
  (macro \texttt{IDT\_ENTRY}) que indica que estamos trabajando con
  una interrupt gate, de tamaño 32 bits, dpl 0 y presente.
\item En el campo \texttt{offset(32:16)} colocamos los primero 2 bytes de la
  dirección donde se encuentra la rutina de atención de la interrupción
  en cuestión.
\end{itemize}

\begin{lstlisting}
  struct str_idt_entry {
    unsigned short offset_0_15;
    unsigned short segsel;
    unsigned char zero;
    unsigned char type:4;
    unsigned char s:1;
    unsigned char dpl:2;
    unsigned char p:1;
    unsigned short offset_16_31;
  } __attribute__((__packed__, aligned (8)));
  
  /* sizeof(struct str_idt_entry) = 8 bytes */
\end{lstlisting}

La macro sencillamente pone en el handler el código de ejecución
definido en \texttt{isr.asm}. Éste sólo imprime en pantalla el tipo de
interrupción.


\subsubsection*{Específicos de este ejercicio}

Para probar el uso de estas rutinas por parte del procesador,
realizamos una división por cero, y efectivamente, se nos imprime en
pantalla el mensaje de \texttt{Divide Error}.


\subsubsection{Ejercicio 3: Iniciar el directorio de página} 

La inicialización del directorio y la tabla de páginas la
completamos sin hacer uso de la funci\'on inicializar\_dir\_kernel
sino que lo hamos directamente en kernel.asm de la siguiente forma:
\begin{itemize}
\item Inicializamos todas las posiciones del Page Directory (PD) en
  0x00000002.
\item Apuntamos la primer página de PD a la page table y seteamos los
  bits de presente y lectura/escritura
\item Realizamos el identity mapping para las \texttt{0x1000} cada
  p\'aginas de la \texttt{PAGE TABLE} y activamos la lectura/escritura
  y el bit de presencia de cada página. (atributos)
\end{itemize}

Para relacionar una entrada en la tabla de páginas con una posicíon de
memoria física, necesitamos dividir la posición por \texttt{0x1000}
(\texttt{4 KB}), que es el tamaño de las páginas. Esto es lo que nos
permite direcccionar hasta \texttt{4 GB} de memoria con la estructura
de directorio+tabla de paginación. esto lo cumplimos al utilizar como
posición inicial del Page Directory a la \texttt{0x27000} y a la del Page Table
\texttt{0x28000}.


\begin{lstlisting}
  struct str_pd_entry {
    unsigned char present:1;
    unsigned char rw:1;
    unsigned char user_supervisor:1;
    unsigned char page_level_write_through:1;
    unsigned char page_level_cache_disable:1;
    unsigned char accessed:1;
    unsigned char ignored:1;
    unsigned char page_size:1;
    unsigned char global:1;
    unsigned char disp:3;
    unsigned int  base_dir:20;
} __attribute__((__packed__));
  /* sizeof(struct str_pd_entry) = 4 bytes */
\end{lstlisting}

\begin{lstlisting}
  struct str_pt_entry {
    unsigned char present:1;
    unsigned char rw:1;
    unsigned char user_supervisor:1;
    unsigned char page_level_write_through:1;
    unsigned char page_level_cache_disable:1;
    unsigned char accessed:1;
    unsigned char dirty_bit:1;
    unsigned char page_table_attr_index:1;
    unsigned char global:1;
    unsigned char disp:3;
    unsigned int  base_dir:20;
  } __attribute__((__packed__));
  /* sizeof(struct str_pt_entry) = 4 bytes */
\end{lstlisting}

Para activar paginación cargamos en el campo base del \texttt{cr3} la
dirección del \PD y seteamos el bit de paginación con \texttt{or cr3,
  0x80000000}.

\subsubsection*{Específicos de este ejercicio}
Testeamos el ejercicio desmapeando la última página
\texttt{0x3FF000}.


\subsubsection{Ejercicio 4: Unidad de Manejo de Memoria}

Escribimos una rutina (\texttt{inicializar\_mmu}) que inicializa las
estructuras globales necesarias para administrar la memoria en el area
libre (un contador de paginas libres):

\begin{itemize}
\item \texttt{PRIMER\_PAG\_LIBR: 0x101000}
\item \texttt{ULTIMA\_PAG\_LIBR: 0x3FE000}
\end{itemize}

Además implementamos algunas funciones para el mapeo de las
páginas:

\begin{itemize} 
\item \texttt{int mmu\_proxima\_pagina\_fisica\_libre()}
\item \texttt{void mmu\_mapear\_pagina (uint virtual, uint cr3, uint fisica, uint attrs)}
\end{itemize}

Esto fue necesario para escribir la rutina
\texttt{mmu\_inicializar\_memoria\_perro} encargada de inicializar un
directorio de páginas y una tabla de páginas para cada una de las
tareas.

\subsubsection*{Específicos de este ejercicio}

Con todo implementado inicializamos la tarea de un perro y la
intercambiamos con el kernel, luego cambiamos de el color del fondo
del primer caracter. Para luego volver a la normalidad, con esto nos
aseguramos el correcto funcionamiento de lo recién implmentado.


\subsection{Ejercicio 5: Interrupciones del Usuario}

Agregamos 3 entradas a la IDT:

\begin{itemize} 
\item \texttt{32} para reloj
\item \texttt{33} para teclado
\item \texttt{0x46} para syscall
\end{itemize}

Las primeras 2 entradas las completamos de la misma forma que hicimos
con las interrupciones del procesador en el punto 3. Es decir, para
completar la entrada de \IDT pusimos como offset el campo
\texttt{offset(15:0)} la direccióon donde se encuentra la rutina de
atención de la interrupción en cuestión, en el selector de segmento
ponemos 0x08 (que corresponde a un segmento de código de nivel 0) y en
los atributos colocamos \texttt{0x8E00} (que indica que estamos
trabajando con una interrupt gate, de tamaño 32 bits, dpl 0 y
presente). La entrada de IDT correspondiente a la interrupcion \texttt{0x46},
la completamos exactamente de la misma forma, pero poniendo el dpl en
3 (o sea attrs en 0xEE00), pues es una interrupcion de tareas.

Luego hay que escribir los handlers de las interrupciones: 

\subsubsection{Reloj}

Esta rutina, por cada tick, llama a la función
\texttt{proximo\_reloj}, que realiza la animación de un cursor
rotando. La función \texttt{proximo\_reloj} está definida en
\texttt{isr.asm}.

Esta rutina entonces es la entrada \texttt{\_isr32} en
\texttt{isr.asm}. Hace los siguientes pasos:

\begin{enumerate}
\item desactivamos interrupciones \texttt{cli}
\item preservamos registros \texttt{pushfd}
\item preservamos EFLAGS \texttt{pushad}
\item comunicamos al PIC que se atendió la interrupción
  \texttt{fin\_intr\_pic1}
\item llamamos a la rutina \texttt{proximo\_reloj}
\item restauramos EFLAGS \texttt{popfd}
\item restuaramos registros \texttt{popad}. 
\item activamos interrupciones \texttt{sti}
\item retornamos \texttt{iret}
\end{enumerate}

En el caso de la rutina de teclado, hacemos esencialmente lo mismo,
pero movemos al registro \texttt{eax} la tecla apretada mediante
\texttt{in al, 0x60}. ponemos \texttt{eax} en la pila y llamamos a la
funcion \texttt{game\_atender\_teclado}. Luego restauramos \texttt{eax}
y su ruta.

La función \texttt{game\_atender\_teclado} es la que se encarga de realizar
alguna acción segun la tecla que fue presionada.

Para la rutina de la \texttt{\_isr46} por ahoa sólo modifica el vaor
de \texttt{eax}, pero luego va a modificarse.

\subsubsection{Ejercicio 6: Tabla de Tareas}

\begin{lstlisting}
  struct str_tss {
    unsigned short  ptl;
    unsigned short  unused0;
    unsigned int    esp0;
    unsigned short  ss0;
    unsigned short  unused1;
    unsigned int    esp1;
    unsigned short  ss1;
    unsigned short  unused2;
    unsigned int    esp2;
    unsigned short  ss2;
    unsigned short  unused3;
    unsigned int    cr3;
    unsigned int    eip;
    unsigned int    eflags;
    unsigned int    eax;
    unsigned int    ecx;
    unsigned int    edx;
    unsigned int    ebx;
    unsigned int    esp;
    unsigned int    ebp;
    unsigned int    esi;
    unsigned int    edi;
    unsigned short  es;
    unsigned short  unused4;
    unsigned short  cs;
    unsigned short  unused5;
    unsigned short  ss;
    unsigned short  unused6;
    unsigned short  ds;
    unsigned short  unused7;
    unsigned short  fs;
    unsigned short  unused8;
    unsigned short  gs;
    unsigned short  unused9;
    unsigned short  ldt;
    unsigned short  unused10;
    unsigned short  dtrap;
    unsigned short  iomap;
  } __attribute__((__packed__, aligned (8)));

  /* sizeof(struct str_tss) = 104 */
\end{lstlisting}


Como primer paso, definimos las entradas para la tarea inicial e idle
en la \GDT dejando algunos valores para completar desde
\texttt{tss.c}:

\begin{itemize}
\item \texttt{base\_0\_15}
\item \texttt{base\_23\_16}
\item \texttt{base\_31\_24}
\item \texttt{limit\_0\_15}
\end{itemize} 

Estos valores se van a definir una vez inicializada la tabla con
\texttt{tss\_inicializar}.

Para inicializar la \TSS de la tarea \texttt{Idle}, realizamos una
función (\texttt{tss\_initializar\_tarea\_idle}) que mapea a dirección
virtual del código de la tarea con la dirección física del código de idle y
completa la entrada \TSS con los siguientes valores:
\begin{itemize}
\item \texttt{cr3} es el del del kernel (no cambiamos el \PD);
\item \texttt{eip} es la posición del código de la tarea en el \PD del
  kernel (porque es el que seteamos en \texttt{cr3}).
\item \texttt{EFLAGS} en \texttt{0x202}, es decir con interrupciones
  en 1 y presente;
\item Como la tarea IDLE corre en nivel 0, se mantienen los registros
  de segmentación del kernel;
\item Se apuntan el esp0, el esp y el ebp a donde se aloja la pila de
  la tarea IDLE
\end{itemize}

La función \texttt{tss\_inicializar\_tarea} tiene como fin completar
la \TSS de una tarea en particular, la que se pasa por parámetro y lo
hace de la siguiente manera:
\begin{itemize}
\item \texttt{cr3} el directorio de páginas de la tarea (a priori no
  tiene por qué ser el del kernel);
\item \texttt{eip} a donde se encuentra la
  tarea en el \PD;
\item \texttt{EFLAGS} nuevamente en \texttt{0x202};
\item Como las tareas corren en nivel 3 y para cumplir con lo
  pedido, se almacenan los siguientes valores para los registros de
  segmentación:
  \begin{itemize}
  \item \texttt{es = ss = ds = fs = gs = 0x33}, que es un
    segmento de datos de nivel 3;
  \item \texttt{cs = 0x2B}, que es un segmento de código de nivel
    3;
  \end{itemize}
\item Tanto \texttt{esp} como \texttt{ebp} se ubican en la dirección de la pila
  de la tarea que es \texttt{TASK\_STACK};
\item El \texttt{esp0} en la dirección física de la pila de nivel 0 de
  la tarea.
\item El \texttt{ss} de nivel 0 en el segmento de datos de nivel 0 ubicado en
  la posicipón \texttt{0x60};
\end{itemize}

En \texttt{tss\_initializar\_tarea\_inicial} no hace falta setear
valores para la \TSS puesto que nunca se va a saltar a esa
tarea, sólo nos interesa setear los valores de la GDT que no pudimos
setear en la inicializaciónn.

Para completar los descriptores anteriormente nombrados,
utilizamos una función que se llama \texttt{tss\_inicializar}, la cual se
encarga de llamar a :
\begin{itemize}
\item \texttt{tss\_initializar\_tarea\_inicial}
\item \texttt{tss\_initializar\_tarea\_idle}
\end{itemize}

Una vez hecho todo eso, para ejecutar la tarea \texttt{Idle}, primero
llamamos a \texttt{inicializar\_scheduler}, que inicializa todas las
estructuras necesarias para arrancar el scheduler. Posteriormente
cargamos el selector de la tarea inicial en un registro, es decir, el
\texttt{0x40} y lo cargamos en el task register. Luego de cargar el
task register, hacemos un cambio de tareas pasando a la tarea idle
mediante un \texttt{0x48:0}. La CPU correrá la tarea idle y cuando se
produzca la próxima interrupción de reloj saltará a la primera tarea.


\subsubsection{Ejercicio 7: Scheduler}

Implementamos la función \texttt{sched\_inicializar} seteando:

Primero para la \texttt{Idle}:
\begin{itemize}
\item \texttt{scheduler.tasks[MAX\_CANT\_TAREAS\_VIVAS].gdt\_index =
    COD\_TAREA\_IDLE\_DIR}
\item \texttt{scheduler.tasks[MAX\_CANT\_TAREAS\_VIVAS].perro = NULL}
\end{itemize}

Luego para cada perro en el arreglo de tareas: 

\begin{itemize}
\item \texttt{scheduler.tasks[i].gdt\_index =
  (GDT\_IDX\_TSS\_BASE\_PERROS\_A + i)}
\item \texttt{scheduler.tasks[i].perro = NULL}
\end{itemize}
y
\begin{itemize}
\item \texttt{scheduler.tasks[MAX\_CANT\_PERROS\_VIVOS + i].gdt\_index =
  (GDT\_IDX\_TSS\_BASE\_PERROS\_B + i)}
\item \texttt{scheduler.tasks[MAX\_CANT\_PERROS\_VIVOS + i].perro = NULL}
\end{itemize}

aquí el índice \texttt{i} corre hasta 8, seteando los 8 perros de cada
jugador.

Y por último, el estado inicial del \texttt{scheduler}, con los
jugadores:

\begin{itemize}
\item \texttt{scheduler.current = 0}
\item \texttt{scheduler.indice\_ultimo\_jugador\_A = 1}
\item \texttt{scheduler.indice\_ultimo\_jugador\_B = 9}
\end{itemize}


Implementamos la función \texttt{sched\_proxima\_a\_ejecutar}, cuya
funcionalidad es hacer el switcheo round-robin de tareas entre los
perros de cada jugador. Intercala las ejecuciones de los perros de
cada jugador a la vez sin repetir el perro en caso de que tengan vivo
mas de uno.

Específico al problema es cómo se guarda la lista de tareas del
scheduler. Lo más importante es tener las rutinas que la manejen
adecuadamente. Básicamente, tenemos que poder:

\begin{itemize}
\item Agregar tarea
\item Remover tarea
\item Siguiente tarea
\end{itemize}

Es natural suponer que la estructura de datos óptima de la lista de
tareas del scheduler va a depender mucho de la naturaleza del SO. En
nuestro caso, la lista de tareas es un array, así que quitar una tarea
es disminuir la cantidad de perros del jugador en cuestión y liberar
la última posición. Agregar una tarea es aumentar la cantidad de
perros del jugador y setear la entrada en la \TSS de acuerdo al tipo
de perro y al nombre del jugador.

Así, \texttt{sched\_proxima\_a\_ejecutar}, va cambiando entre los
jugadores (en caso de que los dos tengan tareas activas) y dentro de
cada jugador toma la acción del siguiente perro.  Cada interrupción de
reloj llama a \texttt{sched\_proxima\_a\_ejecutar} en caso de que haya
un perro vivo y en caso de que no haya tareas en ninguno de los
jugadores, ejecuta la tarea \texttt{Idle}.

Finalmente, los movimientos de los perros requieren privilegios de
kernel para, por ejemplo, pintar el mapa. Como esto no pueden hacerlo
las tareas directamente (no tienen los permisos), hay que hacerlo a
través de una interrupción. En este caso, modificamos la interrupción
de \texttt{syscall} \texttt{0x46} para que comunique al kernel la
acción del perro. Como esta interrupción la tiene que pedir el
usuario, la entrada de \IDT de \texttt{0x46} tiene que tener
\texttt{DPL = 3}. Una vez que se lanza la interrupción, el kernel la
maneja y decide qué hace. Es decir, la interrupción sólo es un
\emph{pedido} al kernel. Éste después decide si le da curso y cómo,
pudiendo incluso desalojar otra tarea.

\subsubsection{Modo Debug}

Tal como se especificó en el enunciado, las excepciones del procesador
activan el modo debug en el juego, en el cual se setea la variable
booleana \texttt{debug}. Así, cuando se produzca la próxima excepción
causada por una tarea se imprime el cartel con el tipo de excepción
que produjo y el valor de cada uno de los registros de la \TSS de esa
tarea y el estado de la pila. Por último, al volver a presionar una
tecla se desactiva el modo Debug y se reanuda el juego desde el estado
anterior.

\newpage




\end{document}
