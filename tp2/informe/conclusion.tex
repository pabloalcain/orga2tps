\indent B\'asicamente, este trabajo pr\'actico nos ha hecho pensar como implementar ciertas funciones de una manera distinta a la que estamos acostumbrados. Adem\'as, nos ha hecho conocer en profundidad el modelo \textit{SIMD}.\\
\indent La programaci\'on vectorial es muy \'util para optimizar funciones. Creemos que nuestras implementaciones muestran claramente la ganancia en performance que hay al programar as\'i. Sin embargo, hay un costo. Los algoritmos son m\'as complejos debido a que el n\'umero de instrucciones aumenta significativamente, se debe pensar cuidadosamente como se cargan los datos de memoria y como se guardan para evitar errores de \textit{Violaci\'on de segmento}. Muchas veces los datos deben ser transformados y reacomodados para poder usarlos con las operaciones \textit{SIMD}.
Tambi\'en hay que analizar bien cuando finalizar los ciclos y si se debe tratar aparte y como se deben tratar los \'ultimos datos, esos que no pudieron ser tratados en el ciclo debido a que no eran suficientes como para poder cargarlos en un registro \texttt{xmmx}.
La otra desventaja que tiene este modo de operaci\'on es que nos ata a una arquitectura espec\'ifica haciendo nuestro programa no portable. Nos ha pasado tener que repensar ciertos partes del c\'odigo por no contar, por ejemplo, con la extensi\'on \texttt{SSE3}.\\
\indent Para cerrar, nos result\'o interesante ver la aplicaci\'on de la programaci\'on vectorial en un tema concreto como el tratamiento de im\'agenes.

